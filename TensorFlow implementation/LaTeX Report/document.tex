%!TeX TS-program = xelatex
%% ARARA spellchain
% arara: xelatex
%% arara: bib2gls
% arara: pythontex: {options: "--runall=true"}
% arara: xelatex
%%% arara: pythontex: {options: "--runall=true"}
%%% arara: xelatex
% arara: split_report


\documentclass[12pt,a4paper]{article}
\usepackage{fontspec} %Включаем возможность настройки шрифтов xelatex
%% Включаем русские буквы в формулах
%%%%%%%%%%%%%%%%%%%%%%%%%%%%%%%%%%%%%%%%%
%\usepackage[paper=A4,pagesize]{typearea}
\usepackage{pdflscape}

\usepackage{amsmath,amssymb}
\usepackage{unicode-math}
\usepackage{polyglossia}
\setmainlanguage{russian}
\setdefaultlanguage[spelling=modern]{russian}
\setotherlanguage{english}
\usepackage[russian]{hyperref}

\setmainfont{Verdana}
\setmathfont{Latin Modern Math}
%\newfontfamily\cyrillicfont{Verdana}[Script=Cyrillic]
\newfontfamily{\cyrillicfonttt}{Courier New}
\setcounter{tocdepth}{3} % включение вложенности в оглавлении до subsubsection

\DeclareSymbolFont{cyrletters}{\encodingdefault}{\familydefault}{m}{it}
\newcommand{\makecyrmathletter}[1]{%
	\begingroup\lccode`a=#1\lowercase{\endgroup
		\Umathcode`a}="0 \csname symcyrletters\endcsname\space #1
}
\count255="409
\loop\ifnum\count255<"44F
\advance\count255 by 1
\makecyrmathletter{\count255}
\repeat
%%%%%%%%%%%%%%%%%%%%%%%%%%%%%%%%%%%%%%%%
\usepackage{chngcntr}
\usepackage{indentfirst}
\def\labelitemi{--} % чтобы в перечислении были -- вместо точки
\usepackage{fancyhdr} %for lfoot
\usepackage{hyperref} %for hyperreference
\usepackage{cprotect} %for hyperreference verbatim
\usepackage{listings} %for inline verbatim

\usepackage{spverbatim}
%\usepackage[russian]{babel} % Включаем поддержку русского языка TODO: Babel для pdflatex, и luatex
% Включаем основным шрифтом Verdana
\usepackage{float} % for [H]
\usepackage{fancybox}
\usepackage{graphicx}
\usepackage{adjustbox}
\usepackage{textcomp} % для значка градуса
\usepackage{calc}
%\usepackage{subfigure} % для вставки коллажей из рисунков
\usepackage[subrefformat=parens,labelformat=parens]{subcaption} % со скобками вокруг ссылки на подрисунок
\renewcommand\thesubfigure{\asbuk{subfigure}} %русские подписи в подрисунках
\usepackage{caption} % Для изменения параметров подписи к рисункам и таблицам 
\usepackage{threeparttable} % Для красиввых подписей к таблицам
\captionsetup[figure]{name={Рисунок}, labelsep={endash}} % Установка стиля подписей к рисункам
\captionsetup[table]{justification=raggedleft,singlelinecheck=off} %установка стиля подписи к таблицам
\renewcommand{\labelenumii}{\arabic{enumi}.\arabic{enumii}.} %убираем сквозную нумерацию рисунков
\usepackage{pythontex} % Поддержка python кода в документе
\usepackage[svgnames,table]{xcolor}
\usepackage[tableposition=above]{caption}
\usepackage{tabulary}
\usepackage{tabularx} %для таблиц переменной ширины
\setlength{\tabcolsep}{2pt}
\usepackage{pifont}
\usepackage{chngpage}
%%tttttttttttttttttttttt
%Tables
\usepackage{booktabs}
\usepackage[table]{xcolor}
\usepackage{widetable}
\usepackage{longtable}
%\usepackage{parskip}[indent=1.5cm] тут непонятная пока неразбериха твориться с стилем параграфа
\usepackage{threeparttablex}
\newcolumntype{R}[1]{>{\raggedleft\let\newline\\\arraybackslash\hspace{0pt}}m{#1}}
\newcolumntype{C}[1]{>{\centering\let\newline\\\arraybackslash\hspace{0pt}}m{#1}}
\newcolumntype{L}[1]{>{\raggedright\let\newline\\\arraybackslash\hspace{0pt}}m{#1}}

\newcolumntype{Y}{>{\centering\let\newline\\\arraybackslash\hspace{0pt}}X}
\newcolumntype{Z}{>{\raggedleft\let\newline\\\arraybackslash\hspace{0pt}}X}

\DeclareCaptionLabelSeparator{tire}{ - }
\captionsetup[table]{labelsep=tire}
%ttttttttttttttttttttttttt
\usepackage[record,% using bib2gls
abbreviations,
]{glossaries-extra}

\GlsXtrLoadResources[
src={abbreviations},% entries in abbrvs.bib
type=abbreviations, % put these entries in the 'abbreviations' list
sort={custom}, % RU first,ru second
sort-rule={< А < Б < В < Г < Д < Е < Ё < Ж < З < И < Й < К < Л < М < Н < О < П < Р < С < Т < Й < Ф < Х < Ц < Ч < Ш < Щ < Э < Ю < Я < A < B < C < D < E < F < G < H < I < J < K < L < M < N < O < P < Q < R < S < T < U < V < W < X < Y < Z}%
]

\usepackage[maxfloats=200]{morefloats} % для решения проблемы большого количества float
\usepackage{placeins} % другой способ решения проблемы кучи float, вставка \FloatBarrier
\input{templateLib/abbrevusualstyle}
% для генерации галочки


%% Вот этот оверкилл нужен был для генерации галки через бешенную библиотеку tikz
\usepackage{tikz}
%\def\checkmark{\tikz\fill[scale=0.4](0,.35) -- (.25,0) -- (1,.7) -- (.25,.15) -- cycle;
\newlength{\leftspace}

%---------------------------------------
%Rotateable ans sizeable pages here
\newlength{\pagewidthAfour} 
\newlength{\pageheightAfour}
\setlength{\pagewidthAfour}{210mm}
\setlength{\pageheightAfour}{297mm}

\newlength{\pagewidthAthree}
\newlength{\pageheightAthree}
\setlength{\pagewidthAthree}{297mm}
\setlength{\pageheightAthree}{420mm}
\newlength{\stockwidth}
\newlength{\stockheight}

\usepackage{geometry}%Установка стандартных размеров полей
%\geometry{height=\pageheightAfour,width=\pageheightAfour,
%	left=3cm,right=1cm,top=2cm,bottom=2cm,nohead,nofoot}

\pdfpagewidth=\pagewidthAfour \pdfpageheight=\pageheightAfour % to enforce?
\paperwidth=\pagewidthAfour \paperheight=\pageheightAfour     % for TikZ
\stockwidth=\pagewidthAfour \stockheight=\pageheightAfour % hyperref (memoir)?!
\usepackage{layouts}
\newcommand{\generatePageLayouts}{%
    % this command must be called after \begin{document}!
    
    % geometry needs layoutwidth - cause it ignores the above paper sizes!
    % layoutwidth=148mm ok, layoutwidth=\paperwidth NOT ok
    % paperwidth gets reset again internally in newgeometry: in log: *geometry* verbose mode: * layout(width,height): (614.295pt,794.96999pt)
    % but by using \stockwidth, which here is just a custom length: * layout(width,height): (421.10078pt,597.50787pt)
    
    \newgeometry{layoutwidth=\pagewidthAfour,layoutheight=\pageheightAfour,left=3cm,right=1cm,top=2cm,bottom=2cm,nofoot,nohead}
    \savegeometry{LayoutPageAfour}
    
    \newgeometry{layoutwidth=\pagewidthAthree,layoutheight=\pageheightAthree,left=3cm,right=1cm,top=2cm,bottom=2cm,footskip=0cm,headheight=0cm}
  
    \savegeometry{LayoutPageAthree}
    
    \newgeometry{layoutwidth=\pageheightAthree,layoutheight=\pagewidthAthree,left=3cm,right=1cm,top=2cm,bottom=2cm,footskip=0cm,headheight=0cm}
    \savegeometry{LayoutPageAthreeLandscape}
}




\newcommand{\switchToLayoutPageAfour}{%
    % doesn't include page sizes; so page size too:
    \pdfpagewidth=\pagewidthAfour \pdfpageheight=\pageheightAfour % for PDF output
    \paperwidth=\pagewidthAfour \paperheight=\pageheightAfour     % for TikZ
    \stockwidth=\pagewidthAfour \stockheight=\pageheightAfour % hyperref (memoir)?!
    \loadgeometry{LayoutPageAfour} % note; \loadgeometry may reset paperwidth/h!
}

\newcommand{\switchToLayoutPageAthree}{%
    % doesn't include page sizes; so page size too:
    \pdfpagewidth=\pagewidthAthree \pdfpageheight=\pageheightAthree % for PDF output
    \paperwidth=\pagewidthAthree \paperheight=\pageheightAthree     % for TikZ
    \stockwidth=\pagewidthAthree \stockheight=\pageheightAthree % hyperref (memoir)?!
    \loadgeometry{LayoutPageAthree} % note; \loadgeometry may reset paperwidth/h!
}
\newcommand{\switchToLayoutPageAthreeLandscape}{%
    % doesn't include page sizes; so page size too:
    \pdfpagewidth=\pageheightAthree \pdfpageheight=\pagewidthAthree % for PDF output
    \paperwidth=\pageheightAthree \paperheight=\pagewidthAthree    % for TikZ
    \stockwidth=\pageheightAthree \stockheight=\pagewidthAthree % hyperref (memoir)?!
    \loadgeometry{LayoutPageAthreeLandscape} % note; \loadgeometry may reset paperwidth/h!
}

%%Addressing overlapping table of contents section number and text problem
\usepackage{tocloft}
\advance\cftsecnumwidth 10pt\relax
\advance\cftsubsecindent 2em\relax
\advance\cftsubsecnumwidth 1em\relax
%%

\begin{pycode}
import glob
import os
import re
import sys
sys.path.append('\\\\192.99.21.6\\VOLUME_8Tb\\EMM Team\\latex_helpers_source\\')
from excel2latexviapython import Excel2latex
import latex_helper_func
from pdf_work import print_hh_formula
insert_image=latex_helper_func.insert_image
insert_image_sequence=latex_helper_func.insert_image_sequence
initialize_helper_func=latex_helper_func.initialize_helper_func
insert_folder_sequence = latex_helper_func.insert_folder_sequence
tex_ref=latex_helper_func.tex_ref
\end{pycode}
%BEGIN_FOLD Список констант для этого документа
\newcommand{\названиеРаботы}
{Разработка модели сверточной нейронной сети, обученную на сгенерированных размеченных данных СКТ}
\newcommand{\идРаботы}
{8888.00.0019-0019/001}
\newcommand{\темаРаботы}
{СКТ}
\newcommand{\номерСлужебки}
{}
\author{И.А.Ниженко}
\newcommand{\ДолжностьАвтораОтчета}{Инженер}

\newcommand{\НашОтдел}{Научно-исследовательский центр суперкомпьютерных технологий}
\newcommand{\НазваниеДокумента}{Служебная записка}
\newcommand{\КомуСлужебкаДолжность}{Заместителю технического директора}
\newcommand{\КомуСлужебкаФИО}{Никитушкину М.В.}
\newcommand{\ОтКогоСлужебкаДолжность}{\ДолжностьАвтораОтчета}
\newcommand{\ОтКогоСлужебкаФИО}{Ниженко И.А.}
\newcommand{\ИсполнительФИО}{Ниженко И.А. \\тел.:    75-39}

%END_FOLD

\usepackage{lipsum}
\usepackage{fp}
\usepackage[utf8]{inputenc}

\usepackage{listings}
\usepackage{xcolor}

\definecolor{codegreen}{rgb}{0,0.6,0}
\definecolor{codegray}{rgb}{0.5,0.5,0.5}
\definecolor{codepurple}{rgb}{0.58,0,0.82}
\definecolor{backcolour}{rgb}{0.95,0.95,0.92}

\lstdefinestyle{mystyle}{
	backgroundcolor=\color{backcolour},   
	commentstyle=\color{codegreen},
	keywordstyle=\color{magenta},
	numberstyle=\tiny\color{codegray},
	stringstyle=\color{codepurple},
	basicstyle=\ttfamily\footnotesize,
	breakatwhitespace=false,         
	breaklines=true,                 
	captionpos=b,                    
	keepspaces=true,                 
	numbers=left,                    
	numbersep=5pt,                  
	showspaces=false,                
	showstringspaces=false,
	showtabs=false,                  
	tabsize=2
}

\lstset{style=mystyle}


\begin{pycode}
	from excel2latexviapython import Excel2latex   
\end{pycode}

\begin{document}
	\newcommand{\названиеКомпании}
{ПАО <<Компания <<Сухой>> <<ОКБ Сухого>>}
\begin{titlepage}
\thisfancypage{%
\setlength{\fboxsep}{2pt}\fbox}{}
\newgeometry{margin=1cm}
Приложение к СЗ: \номерСлужебки  \hfill Гриф: \underline{~~~н/c~~~} \\
\flushright{Экз. №: \underline{\hspace{2cm}}}
\begin{center}
\includegraphics[width=0.1\textwidth]{TemplateLib/logo} \\
{\large \названиеКомпании}
\vspace{4cm}\\
{\textbf{\large Тема: \темаРаботы}} \\
\vspace{0.5cm}
{\Large \textbf{Техническая справка}}\\
\vspace{0.5cm}

по графику \идРаботы \\
\vspace{0.5cm}

{\Large <<\названиеРаботы >> }\\
\vfill
\makeatletter
{
\renewcommand{\arraystretch}{3}
\begin{tabular}{L{.45\textwidth}cR{.45\textwidth}}%>{\raggedright\arrowbackslash}
	Начальник бр. ЭММ НИЦ СКТ & & И.Д. Танненберг \\
	\ДолжностьАвтораОтчета & & И.А. Ниженко \\
	Главный конструктор СКТ & & А.В. Корнев \\ 
\end{tabular} 
}
\makeatother
\vfill

Москва \the\year ~г.
\end{center}

\restoregeometry
\end{titlepage}
	\newpage
	\generatePageLayouts{}
	% start with content
	% start with LayoutPageA (includes switching page size)
	\switchToLayoutPageAfour{}
	%%%%%%%%%%%%%%%%%%%%%%%%%%%%%%%%%%%%%%%%%%%%%%%%%%%%%%%%%%%%%%%%%%%%%%%%%%%%%%%%%%%%%%%%%%%%%%%%%%%%%%%
	%
	\tableofcontents
	\newpage
	%\printunsrtglossary[type=abbreviations, title={Список сокращений}, style=abbrevusualstyle, nonumberlist]
	%\newpage
	%%%%%%%%%%%%%%%%%%%%%%%%%%%%%%%%%%%%%%%%%%%%%%%%%%%%%%%%%%%%%%%%%%%%%%%%%%%%%%%%%%%%%%%%%%%%%%%%%%%%%%%
	\section{Постановка задачи}
	\newpage
	\subsection{Набор данных}
	Как и в случае с любой другой задачей глубокого обучения, первая важная задача - подготовить обучающий набор данных. Набор данных - это топливо, на котором работает любая модель глубокого обучения.
	\newpage
	\subsection{Алгоритм Yolo и его архитектура}
	
	\newpage
	\subsection{Обучение}
	Трансферное обучение - это метод повторного использования уже предварительно обученной модели для решения новой проблемы. В настоящее время он очень популярен в глубоком обучении, потому что может обучать глубокие нейронные сети со сравнительно небольшим объемом данных и за гораздо более короткое время. Это очень полезно, поскольку большинство реальных проблем обычно не имеют миллионов помеченных точек данных для обучения таких сложных моделей.
	\\\\
	При трансферном обучении знания уже обученной модели машинного обучения применяются к другой, но связанной проблеме. Например, если вы обучили простой классификатор предсказать, есть ли на изображении автомобиль, вы можете использовать знания, полученные моделью во время обучения, для распознавания других объектов, например грузовика.
	\\\\
	С трансферным обучением в основном пытаются использовать то, что было изучено в одной задаче, для улучшения обобщения в другой. Переносятся веса, полученные сетью в «задаче A», в новую «задачу B.»
	\\\\
	Общая идея состоит в том, чтобы использовать знания, полученные моделью из задачи с большим количеством доступных помеченных обучающих данных, в новой задаче, в которой не так много данных.
	\\\\
	Трансферное обучение в основном используется в задачах компьютерного зрения и обработки естественного языка, таких как анализ тональности из-за того, что требуется огромная вычислительная мощность.
	\\\\
	Трансферное обучение стало довольно популярным в сочетании с нейронными сетями, которые требуют огромных объемов данных и вычислительной мощности.
	\newpage
	
	\section{Решение задачи}
	
	\pagebreak
	\subsection{Структура программного кода}
	\begin{lstlisting}[language=Python]
	#Blender автоматизация
	
		import bpy, math
		import numpy
		from numpy import genfromtxt
		
		scene = bpy.context.scene
		
		trajectory = genfromtxt('D:\\!IAN_WORK\\NN_vision\\trajectory.csv', delimiter=',')
		
		ob1 = bpy.data.objects["ASP2"]
		ob2 = bpy.data.objects["Box"]
		frame_number = 40
		ob1.animation_data_clear()
		ob2.animation_data_clear()
		
		positions = trajectory[:, :3]
		angles = trajectory[:, 3:]
		
		for angle_i, i in enumerate(positions):
		
			bpy.context.scene.frame_set(frame_number)
		
			ob1.location = i
			ob1.rotation_euler = angles[angle_i]
			ob1.keyframe_insert(data_path='location', index=-1)
			ob1.keyframe_insert(data_path='rotation_euler', index=-1)
		
			ob2.location = i
			ob2.rotation_euler = angles[angle_i]
			ob2.keyframe_insert(data_path='location', index=-1)
			ob2.keyframe_insert(data_path='rotation_euler', index=-1)
		
			frame_number += 0.1
		
		file = open("D:\\!IAN_WORK\\NN_vision\\trajectory.txt", 'w')
		loc = ob1.location
		
		for frames in range(scene.frame_start, scene.frame_end + 1):
		scene.frame_set(frames)
		file.write(str(loc.x) + ', ' + str(loc.y) + ', ' + str(loc.z) + '\n')
		
		file.close()
		
		bpy.context.scene.render.filepath = 'D:\\!IAN_WORK\\NN_vision\\RenderAll\\video\\'
		bpy.ops.render.render(animation=True, use_viewport=True)
		bpy.ops.render.render(write_still=True)
		
		bpy.data.scenes[0].filepath = 'D:\\!IAN_WORK\\NN_vision\\RenderAll\\'
		bpy.ops.render.render(animation = True)
	\end{lstlisting}
	\pagebreak
	\begin{lstlisting}[language=Python]
	#Скрипт для создания коллажа
		
		import cv2
		from PIL import Image,ImageFont,ImageDraw
		import os
		
		def chunks(lst,chunk_size,step):
			for i in range(0,len(lst),chunk_size*step):
				yield  lst[i:i+chunk_size*step:step]
		
		def create_collage(width, height, listofimages):
			cols = 3
			rows = 3
			thumbnail_width = width//cols
			thumbnail_height = height//rows
			size = thumbnail_width, thumbnail_height
			new_im = Image.new('RGB', (width, height),color='white')
			ims = []
			for im in listofimages:
				im.thumbnail(size)
				ims.append(im)
				i = 0
				x = 0
				y = 0
				for row in range(rows):
					for col in range(cols):
						print(i, x, y)
						if i>=len(ims):
							break
						new_im.paste(ims[i], (x, y))
						i += 1
						x += thumbnail_width
						y += thumbnail_height
						x = 0
			return new_im
	
		def create_collage_from_video(videoname,chunk_size=9,step=15):
			basename = os.path.splitext(videoname)[0]
			cap = cv2.VideoCapture(videoname)
			images=[]
			while(cap.isOpened()):
			ret,frame=cap.read()
			if not ret:
				break
			rgb = cv2.cvtColor(frame,cv2.COLOR_BGR2RGB)
			im_pil=Image.fromarray(rgb)
			images.append(im_pil)
			cap.release()
			cv2.destroyAllWindows()
			font = ImageFont.truetype('verdana.ttf',60)
			for index,image in enumerate(images):
				images[index] = image.crop((68,60,1515,914))
				draw = ImageDraw.Draw(images[index])
				draw.text((0,0),u'Кадр: {:.3f} сек.'.format(index),(0,0,0),font=font)
		
		
		image_list = chunks(images[step:],chunk_size,step).__next__()
		create_collage(1920,1200,image_list).save('{}.png'.format(basename.replace('.','_')))
	\end{lstlisting}
	\newpage
	\section{Выводы}

	\newpage


	\switchToLayoutPageAthreeLandscape{}
	\appendix
	\counterwithin{figure}{section}
	\section{Приложение}
%	\begin{figure}[H]
%		\centering 
%		\includegraphics[width=0.9\linewidth]{img/Render/1}
%		\caption{Рендер №1}
%		\label{fig:1}
%	\end{figure}

	\newpage






	{%<<<<<<<<<<<<<<<< Служебная записка!>>>>>>>>>>>>>>>>>>>>>>>>>
		\switchToLayoutPageAfour{}
		\pagebreak
		\pagestyle{fancy}
		\fancyhf{}
		\begin{center}
			\begin{tabular}{L{.48\textwidth}|R{.48\textwidth}}
				
				\НашОтдел & \КомуСлужебкаДолжность \newline \textbf{\КомуСлужебкаФИО}  \\
				\textbf{\номерСлужебки} &  \\
				от 31 декабря 2020 г. & \\
			\end{tabular} 
		\end{center}
		\vfill
		\begin{center}
			\large{\textbf{\НазваниеДокумента}}
		\end{center}
		
		
		%:::::::::Текст служебки::::::::::::::
		Разработана модель сверточной нейронной сети, обученную на сгенерированных данных СКТ, для последующего применения в задаче нейронной сети в соответствии с графиком работ \идРаботы
		\\\\
		Приложение:
		\lstset{
			breaklines=true,
			linewidth=2cm,
			keepspaces=true,
		}
		\begin{itemize}
			\item Модель нейронной сети  %\pyc{print_hh_formula()} 
			расположена на АРМ okb-26341 в директории:
			\cprotect{\href{///\\okb-26341/External share/Finished Works/генератор изображений/.}}{\lstinline[breaklines,linewidth=10cm]{\\okb-26341\External share\Finished Works \генератор изображений\}}
			\end{itemize}
			%:::::::::Текст служебки::::::::::::::
			\vfill
			\begin{center}
				\begin{tabular}{L{.48\textwidth}R{.48\textwidth}}
					
					\ОтКогоСлужебкаДолжность & \ОтКогоСлужебкаФИО \\
					
				\end{tabular} 
			\end{center}
			\vfill
			%{\small \color{gray} Исполнитель: \ИсполнительФИО}\\
			\renewcommand{\headrulewidth}{0pt}
			\lfoot{{\small \color{gray} Исполнитель: \ИсполнительФИО}}
			%\thispagestyle{empty}
			\pagebreak
		}
	\end{document}