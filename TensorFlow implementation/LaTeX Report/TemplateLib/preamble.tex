\usepackage{fontspec} %Включаем возможность настройки шрифтов xelatex
%% Включаем русские буквы в формулах
%%%%%%%%%%%%%%%%%%%%%%%%%%%%%%%%%%%%%%%%%
%\usepackage[paper=A4,pagesize]{typearea}
\usepackage{pdflscape}

\usepackage{amsmath,amssymb}
\usepackage{unicode-math}
\usepackage{polyglossia}
\setmainlanguage{russian}
\setdefaultlanguage[spelling=modern]{russian}
\setotherlanguage{english}
\usepackage[russian]{hyperref}

\setmainfont{Verdana}
\setmathfont{Latin Modern Math}
%\newfontfamily\cyrillicfont{Verdana}[Script=Cyrillic]
\newfontfamily{\cyrillicfonttt}{Courier New}
\setcounter{tocdepth}{3} % включение вложенности в оглавлении до subsubsection

\DeclareSymbolFont{cyrletters}{\encodingdefault}{\familydefault}{m}{it}
\newcommand{\makecyrmathletter}[1]{%
	\begingroup\lccode`a=#1\lowercase{\endgroup
		\Umathcode`a}="0 \csname symcyrletters\endcsname\space #1
}
\count255="409
\loop\ifnum\count255<"44F
\advance\count255 by 1
\makecyrmathletter{\count255}
\repeat
%%%%%%%%%%%%%%%%%%%%%%%%%%%%%%%%%%%%%%%%
\usepackage{chngcntr}
\usepackage{indentfirst}
\def\labelitemi{--} % чтобы в перечислении были -- вместо точки
\usepackage{fancyhdr} %for lfoot
\usepackage{hyperref} %for hyperreference
\usepackage{cprotect} %for hyperreference verbatim
\usepackage{listings} %for inline verbatim

\usepackage{spverbatim}
%\usepackage[russian]{babel} % Включаем поддержку русского языка TODO: Babel для pdflatex, и luatex
% Включаем основным шрифтом Verdana
\usepackage{float} % for [H]
\usepackage{fancybox}
\usepackage{graphicx}
\usepackage{adjustbox}
\usepackage{textcomp} % для значка градуса
\usepackage{calc}
%\usepackage{subfigure} % для вставки коллажей из рисунков
\usepackage[subrefformat=parens,labelformat=parens]{subcaption} % со скобками вокруг ссылки на подрисунок
\renewcommand\thesubfigure{\asbuk{subfigure}} %русские подписи в подрисунках
\usepackage{caption} % Для изменения параметров подписи к рисункам и таблицам 
\usepackage{threeparttable} % Для красиввых подписей к таблицам
\captionsetup[figure]{name={Рисунок}, labelsep={endash}} % Установка стиля подписей к рисункам
\captionsetup[table]{justification=raggedleft,singlelinecheck=off} %установка стиля подписи к таблицам
\renewcommand{\labelenumii}{\arabic{enumi}.\arabic{enumii}.} %убираем сквозную нумерацию рисунков
\usepackage{pythontex} % Поддержка python кода в документе
\usepackage[svgnames,table]{xcolor}
\usepackage[tableposition=above]{caption}
\usepackage{tabulary}
\usepackage{tabularx} %для таблиц переменной ширины
\setlength{\tabcolsep}{2pt}
\usepackage{pifont}
\usepackage{chngpage}
%%tttttttttttttttttttttt
%Tables
\usepackage{booktabs}
\usepackage[table]{xcolor}
\usepackage{widetable}
\usepackage{longtable}
%\usepackage{parskip}[indent=1.5cm] тут непонятная пока неразбериха твориться с стилем параграфа
\usepackage{threeparttablex}
\newcolumntype{R}[1]{>{\raggedleft\let\newline\\\arraybackslash\hspace{0pt}}m{#1}}
\newcolumntype{C}[1]{>{\centering\let\newline\\\arraybackslash\hspace{0pt}}m{#1}}
\newcolumntype{L}[1]{>{\raggedright\let\newline\\\arraybackslash\hspace{0pt}}m{#1}}

\newcolumntype{Y}{>{\centering\let\newline\\\arraybackslash\hspace{0pt}}X}
\newcolumntype{Z}{>{\raggedleft\let\newline\\\arraybackslash\hspace{0pt}}X}

\DeclareCaptionLabelSeparator{tire}{ - }
\captionsetup[table]{labelsep=tire}
%ttttttttttttttttttttttttt
\usepackage[record,% using bib2gls
abbreviations,
]{glossaries-extra}

\GlsXtrLoadResources[
src={abbreviations},% entries in abbrvs.bib
type=abbreviations, % put these entries in the 'abbreviations' list
sort={custom}, % RU first,ru second
sort-rule={< А < Б < В < Г < Д < Е < Ё < Ж < З < И < Й < К < Л < М < Н < О < П < Р < С < Т < Й < Ф < Х < Ц < Ч < Ш < Щ < Э < Ю < Я < A < B < C < D < E < F < G < H < I < J < K < L < M < N < O < P < Q < R < S < T < U < V < W < X < Y < Z}%
]

\usepackage[maxfloats=200]{morefloats} % для решения проблемы большого количества float
\usepackage{placeins} % другой способ решения проблемы кучи float, вставка \FloatBarrier
\input{templateLib/abbrevusualstyle}
% для генерации галочки


%% Вот этот оверкилл нужен был для генерации галки через бешенную библиотеку tikz
\usepackage{tikz}
%\def\checkmark{\tikz\fill[scale=0.4](0,.35) -- (.25,0) -- (1,.7) -- (.25,.15) -- cycle;
\newlength{\leftspace}

%---------------------------------------
%Rotateable ans sizeable pages here
\newlength{\pagewidthAfour} 
\newlength{\pageheightAfour}
\setlength{\pagewidthAfour}{210mm}
\setlength{\pageheightAfour}{297mm}

\newlength{\pagewidthAthree}
\newlength{\pageheightAthree}
\setlength{\pagewidthAthree}{297mm}
\setlength{\pageheightAthree}{420mm}
\newlength{\stockwidth}
\newlength{\stockheight}

\usepackage{geometry}%Установка стандартных размеров полей
%\geometry{height=\pageheightAfour,width=\pageheightAfour,
%	left=3cm,right=1cm,top=2cm,bottom=2cm,nohead,nofoot}

\pdfpagewidth=\pagewidthAfour \pdfpageheight=\pageheightAfour % to enforce?
\paperwidth=\pagewidthAfour \paperheight=\pageheightAfour     % for TikZ
\stockwidth=\pagewidthAfour \stockheight=\pageheightAfour % hyperref (memoir)?!
\usepackage{layouts}
\newcommand{\generatePageLayouts}{%
    % this command must be called after \begin{document}!
    
    % geometry needs layoutwidth - cause it ignores the above paper sizes!
    % layoutwidth=148mm ok, layoutwidth=\paperwidth NOT ok
    % paperwidth gets reset again internally in newgeometry: in log: *geometry* verbose mode: * layout(width,height): (614.295pt,794.96999pt)
    % but by using \stockwidth, which here is just a custom length: * layout(width,height): (421.10078pt,597.50787pt)
    
    \newgeometry{layoutwidth=\pagewidthAfour,layoutheight=\pageheightAfour,left=3cm,right=1cm,top=2cm,bottom=2cm,nofoot,nohead}
    \savegeometry{LayoutPageAfour}
    
    \newgeometry{layoutwidth=\pagewidthAthree,layoutheight=\pageheightAthree,left=3cm,right=1cm,top=2cm,bottom=2cm,footskip=0cm,headheight=0cm}
  
    \savegeometry{LayoutPageAthree}
    
    \newgeometry{layoutwidth=\pageheightAthree,layoutheight=\pagewidthAthree,left=3cm,right=1cm,top=2cm,bottom=2cm,footskip=0cm,headheight=0cm}
    \savegeometry{LayoutPageAthreeLandscape}
}




\newcommand{\switchToLayoutPageAfour}{%
    % doesn't include page sizes; so page size too:
    \pdfpagewidth=\pagewidthAfour \pdfpageheight=\pageheightAfour % for PDF output
    \paperwidth=\pagewidthAfour \paperheight=\pageheightAfour     % for TikZ
    \stockwidth=\pagewidthAfour \stockheight=\pageheightAfour % hyperref (memoir)?!
    \loadgeometry{LayoutPageAfour} % note; \loadgeometry may reset paperwidth/h!
}

\newcommand{\switchToLayoutPageAthree}{%
    % doesn't include page sizes; so page size too:
    \pdfpagewidth=\pagewidthAthree \pdfpageheight=\pageheightAthree % for PDF output
    \paperwidth=\pagewidthAthree \paperheight=\pageheightAthree     % for TikZ
    \stockwidth=\pagewidthAthree \stockheight=\pageheightAthree % hyperref (memoir)?!
    \loadgeometry{LayoutPageAthree} % note; \loadgeometry may reset paperwidth/h!
}
\newcommand{\switchToLayoutPageAthreeLandscape}{%
    % doesn't include page sizes; so page size too:
    \pdfpagewidth=\pageheightAthree \pdfpageheight=\pagewidthAthree % for PDF output
    \paperwidth=\pageheightAthree \paperheight=\pagewidthAthree    % for TikZ
    \stockwidth=\pageheightAthree \stockheight=\pagewidthAthree % hyperref (memoir)?!
    \loadgeometry{LayoutPageAthreeLandscape} % note; \loadgeometry may reset paperwidth/h!
}

%%Addressing overlapping table of contents section number and text problem
\usepackage{tocloft}
\advance\cftsecnumwidth 10pt\relax
\advance\cftsubsecindent 2em\relax
\advance\cftsubsecnumwidth 1em\relax
%%

\begin{pycode}
import glob
import os
import re
import sys
sys.path.append('\\\\192.99.21.6\\VOLUME_8Tb\\EMM Team\\latex_helpers_source\\')
from excel2latexviapython import Excel2latex
import latex_helper_func
from pdf_work import print_hh_formula
insert_image=latex_helper_func.insert_image
insert_image_sequence=latex_helper_func.insert_image_sequence
initialize_helper_func=latex_helper_func.initialize_helper_func
insert_folder_sequence = latex_helper_func.insert_folder_sequence
tex_ref=latex_helper_func.tex_ref
\end{pycode}